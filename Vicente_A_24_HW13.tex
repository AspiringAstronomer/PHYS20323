%%%%%%%%%%%%%%%%%%%%%%%%%%%%%%%%%%%%%%%%%%%%%%%%%%%%%%%%%%%%
%%%%%%%%%%%%%%%%%%%%%%%%%%%%%%%%%%%%%%%%%%%%%%%%%%%%%%%%%%%%
%%%%%%%%%%%%%%%%%%%%%%%%%%%%%%%%%%%%%%%%%%%%%%%%%%%%%%%%%%%%
%%%%%%%%%%%%%%%%%%%%%%%%%%%%%%%%%%%%%%%%%%%%%%%%%%%%%%%%%%%%
%%%%%%%%%%%%%%%%%%%%%%%%%%%%%%%%%%%%%%%%%%%%%%%%%%%%%%%%%%%%
\documentclass[12pt]{article}
\usepackage{epsfig}
\usepackage{times}
\usepackage{fancyhdr}
\usepackage{pslatex}
\usepackage{amsmath}
\usepackage{mathrsfs}
\usepackage[dvipsnames]{xcolor}
\usepackage[hidelinks]{hyperref}%renewcommand{\topfraction}{1.0}
\renewcommand{\topfraction}{1.0}
\renewcommand{\bottomfraction}{1.0}
\renewcommand{\textfraction}{0.0}
\setlength {\textwidth}{6.6in}
\hoffset=-1.0in
\oddsidemargin=1.00in
\marginparsep=0.0in
\marginparwidth=0.0in                                                                               
\setlength {\textheight}{9.0in}
\voffset=-1.00in
\topmargin=1.0in
\headheight=0.0in
\headsep=0.00in
\footskip=0.50in                                         
\setcounter{page}{1}
\begin{document}
\def\pos{\medskip\quad}
\def\subpos{\smallskip \qquad}
\newfont{\nice}{cmr12 scaled 1250}
\newfont{\name}{cmr12 scaled 1080}
\newfont{\swell}{cmbx12 scaled 800}
%%%%%%%%%%%%%%%%%%%%%%%%%%%%%%%%%%%%%%%%%%%%%%%%%%%%%%%%%%%%
%     DO NOT CHANGE ANYTHING ABOVE THIS LINE
%%%%%%%%%%%%%%%%%%%%%%%%%%%%%%%%%%%%%%%%%%%%%%%%%%%%%%%%%%%%
%     DO NOT CHANGE ANYTHING ABOVE THIS LINE
%%%%%%%%%%%%%%%%%%%%%%%%%%%%%%%%%%%%%%%%%%%%%%%%%%%%%%%%%%%%
%     DO NOT CHANGE ANYTHING ABOVE THIS LINE
%%%%%%%%%%%%%%%%%%%%%%%%%%%%%%%%%%%%%%%%%%%%%%%%%%%%%%%%%%%%
\begin{center}
{\large
PHYSICS  X0323: Scientific Analysis \& Modeling - Fall 2024
}\\
%%%%%%%%%%%%%%%%%%%%%%%%%%%%%%%%%%%%%%%%%%%%%%%%%%%%%%%%%%%%
{\large {by:Anahid Vicente}\\\vskip0.25in
%%%%%%%%%%%%%%%%%%%%%%%%%%%%%%%%%%%%%%%%%%%%%%%%%%%%%%%%%%%%
\end{center}
%%%%%%%%%%%%%%%%%%%%%%%%%%%%%%%%%%%%%%%%%%%%%%%%%%%%%%%%%%%%
% Section Heading
%%%%%%%%%%%%%%%%%%%%%%%%%%%%%%%%%%%%%%%%%%%%%%%%%%%%%%%%%%%%
\noindent {\bf } \\

%%%%%%%%%%%%%%%%%%%%%%%%%%%%%%%%%%%%%%%%%%%%%%%%%%%%%%%%%%%%

\noindent {\bf 1: An electron is found to be in the spin state (in the z-basis):\\
(a) (5 points) Determine the possible values of A such as that the state is normalized.\\
(b) (5 points) Find the expectation values of the operators,\textcolor{color}{text} \langle S_x \rangle \),  \( \langle S_y \rangle \), \( \langle S_z \rangle \), and \( \langle \vec{S}^2 \rangle \).
\[
\chi = A \begin{pmatrix} 3i \\ 4 \end{pmatrix}
\]

\noindent \textbf{The Matrix representation in the z-basis for the components of electron spin operators are given by:} \\
\[
S_x = \frac{\hbar}{2} \begin{pmatrix} 0 & 1 \\ 1 & 0 \end{pmatrix}, \quad
S_y = \frac{\hbar}{2} \begin{pmatrix} 0 & -i \\ i & 0 \end{pmatrix}, \quad
S_z = \frac{\hbar}{2} \begin{pmatrix} 1 & 0 \\ 0 & -1 \end{pmatrix}
\]
\noindent {\bf 2. The average electrostatic field in the earth’s atmosphere in fair weather is approximately given:}
\[
\mathbf{E} = E_0 \left( A e^{-\alpha z} + B e^{-\beta z} \right) \hat{z}
\]
Where A, B, $\alpha$, $\beta$ are positive constants and z is the height above the (locally flat) earth surface.\\
(a) (5 points) Find the average charge density in the atmosphere as a function of height\\
(b) (5 points) Find the electric potential as a function height above the earth\\

\noindent{\bf 3. The following questions refer to stars in the Table below.\\
Note: There may be multiple answers.\\




\begin{tabular}{|c|c|c|c|c|c|}
\hline
\textbf{Name} & \textbf{Mass} & \textbf{Luminosity} & \textbf{Lifetime} & \textbf{Temperature} & \textbf{Radius} \\ \hline
 $\beta$ Cyg. & 1.3 $M_\odot$ & 3.5 $L_\odot$ &  &  &  \\ \hline
 $\alpha$ Cen.& 1.0 $M_\odot$ &  &  &  &1 $R_\odot$  \\ \hline
 $\eta$ Car.&60.$M\odot$ &60.$M\dot$ &$8.0 \times 10^5$  & 20,000 K &  \\ \hline
 $\epsilon$ Eri.& 6.0$M\odot$  &  & &  &  \\ \hline
 $\delta$ Scu. & 2.0$M\odot$  &  &$5.0 \times 10^8$   &  &2 $R_\odot$  \\ \hline
 $\gamma$ Del. & 0.7$M\odot$  &  &$4.5 \times 10^10$  &5000 K  &  \\ \hline
\end{tabular}

\textbf{(a)} (4 points) Which of these stars will produce a planetary nebula?\\

\textbf{(b)} (4 points) Elements heavier than Carbon will be produced in which stars?\\





Latex Example
\end{document}





\
\( A \) & \( \frac{1}{5} \) \\ \hline
\( S_x \) & \( \frac{\hbar}{2} \begin{pmatrix} 0 & 1 \\ 1 & 0 \end{pmatrix} \) \\ \hline
\( S_y \) & \( \frac{\hbar}{2} \begin{pmatrix} 0 & -i \\ i & 0 \end{pmatrix} \) \\ \hline
\( S_z \) & \( \frac{\hbar}{2} \begin{pmatrix} 1 & 0 \\ 0 & -1 \end{pmatrix} \) \\ \hline
\( \hbar \) & Planck's constant divided by \( 2\pi \) \\ \hline
\end{tabular}
\end{center}
 








\end{document}



